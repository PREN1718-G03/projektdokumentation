\documentclass[a4paper]{report}

% Use swiss german letters
\usepackage[utf8]{inputenc}

% Language: german
\usepackage[ngerman]{babel}

% Fancy Figures
\usepackage{graphicx}

% Use Times
\usepackage{mathptmx}

% Display the Bibliography in the TOC
\usepackage{tocbibind}

% Better lists
\usepackage{enumitem}

% Use biblatex
\usepackage[style=apa,backend=biber,citestyle=authoryear]{biblatex} 

% Tell BibLatex to use the ngerman language mapping
\DeclareLanguageMapping{ngerman}{ngerman-apa}

% Define the bibliography file
\addbibresource{bibliography.bib}

% To let LaTeX handle "
\usepackage[autostyle=true, german=quotes]{csquotes}

% Titlepage
\newcommand*{\titleAP}{\begingroup % Create the command for including the title page in the document
	\centering
	\vspace*{\baselineskip} % Whitespace at the top of the page
	
	{\Large FirstName LastName}\\[0.167\textheight] % Author name
	
	{\Huge\bfseries Projektdokumentation PREN Gruppe03}\\[\baselineskip]
	
	%TODO review subtitle
	{\Large \textit{Subtitle}}\\
	\today
	
	\vspace*{3\baselineskip} % Whitespace at the bottom of the page
	\endgroup}

% Define the path were images are found
\graphicspath{{./img/}}

\begin{document}

\titleAP

\newpage

\begin{abstract}
	Hier würde man das Abstract oder Management Summary schreiben.
\end{abstract}

\tableofcontents

\newpage

\chapter{Einleitung}
\label{ch:Intro}

\section{Präambel}
Eine Einleitung in ein \LaTeX Dokument. 
Dieses ist im Moment auf Github gehosted, Git ist ein Codeverwaltungsprogramm welches von vielen Programmierern und OpenSource Projekten benutzt wird. \parencite{Git2017}
Ein leichtverständliches Buch wurde von \citeauthor{Chacon2016} geschrieben und ist frei verfügbar.

\chapter{Projektorganisation}

\section{Teamübersicht}

\newpage

\printbibliography

\end{document}