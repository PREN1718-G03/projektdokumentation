\documentclass[a4paper]{report}

% Use swiss german letters
\usepackage[utf8]{inputenc}

% Language: german
\usepackage[ngerman]{babel}

% Fancy Figures
\usepackage{graphicx}

% Use Times
\usepackage{mathptmx}

% Display the Bibliography in the TOC
\usepackage{tocbibind}

% Better lists
\usepackage{enumitem}

% We want SI units!!!
\usepackage{siunitx}

% Use biblatex
\usepackage[style=apa,backend=biber,citestyle=authoryear]{biblatex} 

% Tell BibLatex to use the ngerman language mapping
\DeclareLanguageMapping{ngerman}{ngerman-apa}

% Define the bibliography file
\addbibresource{bibliography.bib}

% To let LaTeX handle "
\usepackage[autostyle=true, german=quotes]{csquotes}

% Rename the Abstract to Management Summary
\addto\captionsngerman{\renewcommand{\abstractname}{Management Summary}}

% Titlepage
\newcommand*{\titleAP}{\begingroup % Create the command for including the title page in the document
	\centering
	\vspace*{\baselineskip} % Whitespace at the top of the page
	
	{\Large FirstName LastName}\\[0.167\textheight] % Author name
	
	{\Huge\bfseries Projektdokumentation PREN Gruppe03}\\[\baselineskip]
	
	%TODO review subtitle
	{\Large \textit{Subtitle}}\\
	\today
	
	\vspace*{3\baselineskip} % Whitespace at the bottom of the page
	\endgroup}

% Define the path were images are found
\graphicspath{{./img/}}

\begin{document}

\titleAP

\newpage

\chapter*{Redlichkeitserklärung}
% TODO Soll eig. kein Chapter sein.. und auch nicht in table of content auftauchen.
% Comments und TODOs kann man mit % einfügen - TeXStudio erkennt diese auch
% Das das Kapitel nicht im ToC erscheint eunfach ein \chapter* benutzen ;-)

\newpage

\begin{abstract}
	Hier würde man das Abstract oder Management Summary schreiben.
\end{abstract}

\tableofcontents

\newpage

\chapter{Einleitung}
\label{ch:Intro}

\chapter{Projektorganisation}

\section{Teamübersicht}

\begin{tabular}{|l|l|}
	\hline 
	\textbf{Name} & \textbf{Studium} \\ 
	\hline 
	Pascal Baumann & Informatik \\ 
	\hline 
	Basil Bachmann & Maschinenbau \\ 
	\hline 
	David Craven & Elektrotechnik \\ 
	\hline 
	Victor Guntern & Maschinenbau \\ 
	\hline
	Markus Kempf & Maschinenbau \\ 
	\hline  
	Eve Meier & Informatik \\ 
	\hline 
	Jan Odermatt & Elektrotechnik \\
	\hline
	Simon Rohrer & Maschinenbau \\
	\hline
\end{tabular} 

\section{Projektrollen}

\begin{tabular}{|l|l|l|}
	\hline 
	\textbf{Rolle} & \textbf{Aufgaben} & \textbf{Teammitglied} \\ 
	\hline 
	Projektleiter & Gesamtübersicht des Projektes halten  & Eve \\
	& Überprüfen ob Vorgaben eingehalten werden & \\
	& Teammeetings organisieren & \\
	& Informationsaustausch sicherstellen & \\
	& Kostenverwaltung & \\ 
	\hline 
	Projektplaner & Aktualisieren des Terminplanes & Markus\\
	& Rahmenplanung und Meilensteineinhaltung & \\
	& Pflege Taskboard & \\ 
	\hline 
	Verantwortlicher& Meilensteinabgabe zusammenstellen & Pascal \\
	Dokumentation& Protokolle führen & \\
	& Unterstützung und Pflege LaTeX &  \\ 
	\hline 
	Fachverantwortliche & Projektstand und Feedback & I Pascal \\
	& Ansprechperson bei Fragen & ET Jan\\
	& Aktualisierung Risikomanagement & M Markus\\
	& Koordination Versuche und Rechere & ?\\ 
	\hline 
\end{tabular} 

\section{Tools}
\begin{tabular}{|l|l|}
	\hline 
	\textbf{Aufgabe} & \textbf{Hilfsmittel} \\ 
	\hline 
	Dokumente und Dokumentation & LaTeX / MiKTeX / GitHub \\
	\hline
	Quellen & Mendeley \\
	\hline
	Dateiablage für Teamaustausch & Dropbox \\
	\hline
	Dateiablage für Abgabe & Ilias \\
	\hline
	Projekt- und Budgetplan & MS Excel 2016 \\
	\hline
	Kommunikation Team & WhatsApp Gruppe od. HSLU Mailadresse\\
	\hline
	Aufgabenverwaltung & SCRUM-angelehntes Board, physisch, Teaminsel\\
	\hline
\end{tabular}

\section{Wochenplan}
\begin{tabular}{|l|l|}
	\hline 
	\textbf{Tag} & \textbf{Beschreibung} \\
	\hline
	DO 08:30 & Alle Mitglieder sind in der Teaminsel \\
	\hline
	DO 08:30-09:00 & Besprechung Team-intern erledigte Aufgaben \\
	& Fragen, weiteres Vorgehen \\
	\hline
	DO 09:00-10:00 & Arbeiten im Team od. selbständig \\
	\hline
	DO 10:00-10:20 & Pause \\
	\hline
	DO 10:20-12:00 & Arbeiten im Team od. selbstständig \\
	\hline
	FR 08:30 & Alle Mitglieder sind in der Teaminsel \\
	\hline
	FR 08:30-09:00 & Besprechung Team-intern, Vorbereitung Meeting mit Dozent \\
	\hline
	FR 09:00-09:30& Besprechung mit Dozent \\
	\hline
	FR 09:30-10:00 & Arbeiten im Team od. selbstständig \\
	\hline
	FR 10:00-10:20 & Pause \\
	\hline
	FR 10:20-11:00 & Arbeiten im Team od. selbstständig \\
	\hline
	FR 11:00-11:30 & Kurzbesprechung, Taskboard aktualisieren \\
	\hline
	FR 11:30-12:00 & Arbeiten im Team od. selbstständig (freiwillig)\\
	\hline
\end{tabular}

\section{Projektplan}
TODO Projektplan einfügen zusätzlich Terminplan mit Arbeitspaketen und Meilensteinen - Rahmenplan

\section{Budgetplan}
Für den Bau der Teilfunktionsmuster in PREN1 dürfen maximal CHF 200.- ausgegeben werden.

\begin{tabular}{|l|l|l|l|}
	\hline
	\textbf{Artikel} & \textbf{Anzahl} & \textbf{Preis/Stk.} & \textbf{Total} \\
	\hline
	Artikel 1 & 2 & 11.50 & 23 \\
	\hline
	\textbf{Total} & & & 23 \\
	\hline
\end{tabular}
	

\chapter{Anforderungen}
\section{Projektanforderungen}
\begin{tabular}{|l|l|l|}
	\hline 
	\textbf{Nr.} & \textbf{Bezeichnung} & \textbf{Beschreibung} \\ 
	\hline 
	1.1 & Projektabgabe & Dezember 2017 \\ 
	\hline 
	1.2 & Eigenleistung & Systemkomponenten können zugekauft werden \\ 
	\hline
	1.3 & Interdisziplinarität & Disziplinen / Abteilungen arbeiten zusammen \\
	\hline
	1.4 & Lieferantenwahl & Für Sammelbestellungen gem. Kapitel 4.5 \\
	& & der Aufgabenstellung \\
	& & Wird Material vom Team selbst gekauft,\\
	& & können die Kosten zurückgefordert werden \\ 
	\hline 
	1.5 & Budget f. PREN & max. 500.- CHF \\ 
	\hline 
	1.6 & Teilbudget PREN1 & max. 200.- CHF \\ 
	\hline 
	1.7 & 3D-Drucker Laufzeit & max. 25 \\ 
	\hline 
	1.8 & Lasergerät Laufzeit & max. 1 \\ 
	\hline 
	1.9 & Stunden ET-Werkstattpersonal & max. 10 \\ 
	\hline 
	1.10 & Stunden M-Werkstattpersonal & max. 10 \\ 
	\hline 
	1.11 & "Gesponsorte"\ Komponenten & Werden mit einem realistischen Preis\\
	& & in die Kostenrechnung einbezogen \\
	\hline
\end{tabular} 

\section{Plattform}
\begin{tabular}{|l|l|l|}
	\hline 
	\textbf{Nr.} & \textbf{Bezeichnung} & \textbf{Beschreibung} \\
	\hline 
	2.1 & Gesamtlänge & 350 $\pm$ 2cm \\ 
	\hline 
	2.2 & Masten Abstand & Abstand zwischen den Masten 350cm $\pm$ 2cm \\
	\hline
	2.3 & Masten Masse & TODO Quadratisch? 10cm $\pm$ 1 cm \\
	\hline
	2.4 & Drahtseil & Verzinkter Stahl, Durchmesser 3mm \\ 
	\hline 
	2.5 & Seilspannung & Via Umlenkrollen durch ein Gewicht\\
	& & mit einer Masse von 15kg \\ 
	\hline 
	2.6 & Winkel des Seiles & TODO \\
	\hline
	2.7 & Grundplatte & Spanplatte roh oder grau gestrichen.\\
	& & Mit Farbresten / vorstehenden Schrauben\\
	& & und Nahtstellen ist zu rechnen \\
	\hline
	2.8 & Startfeld & 50cm $\pm$ 2cm, Quadratisch \\ 
	\hline 
	2.9 & Zielplatte & TODO Gesamtmass? \\ 
	\hline 
	2.10 & Zielplatte Aussehen & TODO Wie viele konzentrische Bereiche?\\
	& & Der innerste, helle Bereich ist quadratisch\\
	& & und hat eine Seitenlänge von 6cm $\pm$ 2mm. \\
	& & Jeder daran anschliessende konzentrische Bereich\\
	& & hat eine Breite von 2.5cm $\pm$ 2mm. \\
	& & Die Bereiche sind abwechslungsweise hell und dunkel \\ 
	\hline 
	2.11 & Zielplatte Position & Der Absetzbereich verläuft unterhalb des Seiles und ist 2cm breit.\\
	& & Die Zielplatte kann bis zum Startsignal verschoben werden.\\
	& & Befindet sich aber immer im Absetzbereich\\
	& & (siehe Abbildung 1 der Aufgabenstellung) \\
	\hline
	2.12 & Start- und Zielplatte & TODO Matt? Glänzend? \\
	\hline
	2.13 & Hindernisse & Auf der gesamten Plattform können Hindernisse stehen.\\
	& & Im Umkreis von mindestens 10cm\\
	& & um die Startposition des Ladegutes\\
	& & und um die Zielplatte herum sind keine Hindernisse \\
	\hline
	2.14 & Hindernisse Höhe & Die Hindernisse haben eine maximale Höhe von 20cm. \\
	\hline
\end{tabular} 

\section{Laufkatze}
\begin{tabular}{|l|l|l|}
	\hline 
	\textbf{Nr.} & \textbf{Bezeichnung} & \textbf{Beschreibung} \\
	\hline 
	3.1 & Steuerung & Autonom \\
	\hline
	3.2 & Inbetriebnahme & Darf max. 2min dauern \\
	\hline
	3.3 & Startsignal & Darf per Kopfdruck gesendet werden \\
	\hline
	3.4 & Geschwindigkeit & Um die Aufgabe zu bewältigen steht der Laufkatze\\
	& & ein Zeitfensters von 4min zur Verfügung. \\
	\hline
	3.5 & Aussendimensionen & Die Laufkatze darf in ihrer Projektion\\
	& & das Startfeld nicht überschreiten.\\
	& & 50cm $\pm$ 2cm x 50cm $\pm$ 2cm \\
	\hline
	3.6 & Bauart & Sämtliche Sensorik muss auf dem Gerät selbst montiert sein. \\
	\hline
	3.7 & Fahrweise & Das Gerät darf nur das Drahtseil und den zweiten Masten berühren.\\
	& & Die gesamte Plattform, insbesondere Drahtseil,\\
	& & die Last und die Zielplatte dürfen nicht beschädigt\\
	& & oder sonst irgendwie verändert werden.\\
	& & Es ist beispielsweise nicht erlaubt, Navigationshilfen anzubringen. \\
	\hline
	3.8 & Ladegut & Das Gerät muss ein Ladegut transportieren können. \\
	\hline
	3.9 & Zielerkennung & Das Erkennen der Zielplatte muss selbstständig erfolgen\\
	\hline 
\end{tabular}

\section{Ladegut}
\begin{tabular}{|l|l|l|}
	\hline 
	\textbf{Nr.} & \textbf{Bezeichnung} & \textbf{Beschreibung} \\
	\hline
	4.1 & Material & Holz \\
	\hline
	4.2 & Dimensionen & Seitenlänge 5cm $\pm$ 0.5cm \\
	\hline
	4.3 & Gewicht & TODO \\
	\hline
	4.4 & Aufnahme & Metallischer, magnetischer Hacken oben in der Mitte des Würfels.\\
	& & Innendurchmesser des Hakens ist $\pm$ 1cm. \\
	\hline
	4.5 & Hindernisse & Das Ladegut darf Hindernisse nicht berühren. \\
	\hline
	4.6 & Position & Die Position des Ladegutes muss in Echtzeit angezeigt werden,\\
	& & dass der Schiedsrichter jederzeit die angezeigten Werte\\
	& & gut erkennen kann.\\
	& & TODO auf Gerät od. Extern? \\
	\hline
	4.7 & Positionsbestimmung & Die Mitte des Bodens des Ladegutes wird verwendet.\\
	& & Die Position muss in x- und z-Richtung bestimmt werden.\\
	& & Der Nullpunkt des zu verwendenden Koordinatensystems\\
	& & ist in Abbildung 1 der Aufgabenstellung definiert. \\
	\hline
	4.8 & Absetzen & Das Ladegut muss innerhalb\\
	& & des Zielbereiches automatisch abgesetzt werden. \\
	\hline
	4.9 & Zielbereich & Der Zielbereich muss automatisch erkennt werden\\
	\hline
\end{tabular}

\section{Umfeld}
\begin{tabular}{|l|l|l|}
	\hline 
	\textbf{Nr.} & \textbf{Bezeichnung} & \textbf{Beschreibung} \\
	\hline
	5.1 & Licht & TODO \\
	\hline
	5.2 & Temperaturen & Bei Lagerung und Betrieb Zimmertemperatur 15-\SI{20}{\degreeCelsius}\\
	\hline
\end{tabular}

\chapter{Risikomanagement}
In diesem Kapitel werden mögliche Risiken während des Projektverlaufes aufgelistet. Dabei werden Projektrisiken nummeriert. Ihre Eintrittswahrscheinlichkeit und ihr Schadensausmass wird eingeschätzt. Besteht ein grosses Risiko, werden zusätzlich Massnahmen definiert. 

\section{Definitionen}
TODO Eintrittswahrscheinlichkeit und Schadensausmass definieren. Hier die Einteilung aus der Präsentation zu  Projektmanagement:

%TODO Stimmen dieses vertical Spacers noch? 
\vspace{1em}
\noindent
Eintrittswahrscheinlichkeit: 

%TODO Stimmen dieses vertical Spacers noch?
\vspace{1em}
\noindent
\begin{tabular}{|l|l|l|}
	\hline
	\textbf{Stufe} & \textbf{Bezeichnung} & \textbf{Beschreibung} \\
	\hline
	1 & unvorstellbar & \\
	\hline
	2 & unwahrscheinlich & \\
	\hline
	3 & vorstellbar & \\
	\hline
	4 & wahrscheinlich & \\
	\hline
	5 & häufig & \\
	\hline
\end{tabular}

%TODO Stimmen dieses vertical Spacers noch?
\vspace{1em}
\noindent
Schadensausmass:

%TODO Stimmen dieses vertical Spacers noch?
\vspace{1em}
\noindent
\begin{tabular}{|l|l|l|}
	\hline
	\textbf{Stufe} & \textbf{Bezeichnung} & \textbf{Beschreibung} \\
	\hline
	1 & unwesentlich & \\
	\hline
	2 & geringfügig & \\
	\hline
	3 & mittelmässig & \\
	\hline
	4 & kritisch & \\
	\hline
	5 & katastrophal & \\
	\hline
\end{tabular}
	
	
\section{Risikokatalog}

\begin{tabular}{|l|l|l|l|}
	\hline 
	\textbf{Nr.} & \textbf{Risiko} & \textbf{Wahrscheinlichkeit} & \textbf{Schadensausmass} \\
	\hline
	x & Teammitglied fällt bis zu 2 Wochen aus & & \\
	\hline 
	x & Teammitglied fällt komplett aus & & \\
	\hline
	x & Kommunikation im Team schwierig & & \\
	\hline
	x & Datenverlust & & \\
	\hline 
	x & Testattermin kann nicht eingehalten werden & & \\
	\hline
	x & Mit Dokumentation wird zu spät begonnen & & \\
	\hline
	x & Anforderungen werden nicht & & \\
	& richtig verstanden & & \\
	\hline
	x & Anforderungen werden nicht eingehalten & & \\
	\hline
	x & Konzeptänderung in PREN2 & & \\
	\hline
\end{tabular}

\vspace{1em}
\noindent
\begin{tabular}{|l|l|l|l|}
	\hline 
	\textbf{Nr.} & \textbf{Risiko} & \textbf{Wahrscheinlichkeit} & \textbf{Schadensausmass} \\
	\hline
	x & Fertigungszeiten für Teile unterschätzt & & \\
	\hline
	x & Mehr Maschinenlaufzeiten benötigt & &  \\
	\hline
	x & Aufbau vor Start dauert länger als 2 Minuten & & \\
	\hline
	x & Startsignal wird nicht erkannt & & \\
	\hline
	x & Laufkatze bleibt während Fahrt stecken & &\\
	& Seilspannung / Antrieb & &\\
	\hline
	x & Laufkatze bleibt nach Anhalten & &\\
	& um Ladegut zu heben stecken& &\\
	& und Anfahren geht nicht mehr & & \\
	\hline
	x & Kameraqualität zu schwach für Anwendung & & \\
	\hline
	x & Störfaktor Lichteinflüsse von Aussen & & \\
	\hline
	x & Ladegut wird nicht aufgenommen /& &\\
	& rutscht ab / zu schwer & & \\
	\hline
	x & Ladegut berührt Hindernisse & & \\
	\hline
	x & Ladegut trifft beim absetzen Zielplatte nicht & & \\
	\hline
	x & Koordinatensystem kann nicht & &\\
	& abgebildet werden & & \\
	\hline
	x & Position des Ladeguts wird falsch& &\\
	&  oder gar nicht angezeigt & & \\
	\hline
\end{tabular}
\newpage

\section{Risikobewertung}
TODO tolles farbiges Matrixteil yeah
 
\section{Massnahmen}
TODO Bei Risiken im Roten und Gelben Bereich Massnahmen definieren

% TODO Stimmen diese vertikalen Spacers noch?
\vspace{1em}
\noindent
\begin{tabular}{|l|l|}
	\hline 
	\textbf{Risiko Nr.} & \textbf{Massnahme} \\
	\hline
	x & Bla \\
	\hline
\end{tabular}

\chapter{Technologierechere}
% TODO Sammlung verschiedener Technologien: Plattformen, Programmiersprachen und Bibliotheken, Materialien, Bauart, Antrieb, Greifer, Sensoren, Visualisierung Lastposition.. etc.
% TODO Vor- und Nachteile, Quelle, Wertung zur Evaluation... 

\section{Plattformen}

\subsection{Informatik}
\vspace{1em}
\noindent
\begin{tabular}{|l|l|}
	\hline 
	\textbf{Beschreibung} & \textbf{Quelle} \\
	\hline
	TinkerBoard & http://www.trustedreviews.com/reviews/asus-tinker-board \\
	\hline
	Raspberry Pi 3 Model B & https://www.raspberrypi.org/products/raspberry-pi-3-model-b/\\
	&\#buy-now-modal \\
	\hline
	Arduino Uno R3 & https://store.arduino.cc/arduino-uno-rev3\\
	\hline
	Banana PI M2 Berry & http://www.banana-pi.org/m2ub.html \\
	\hline
	BeagleBone Black Rev & https://beagleboard.org/black/\\
	\hline
\end{tabular}

\vspace{1em}
Sowohl bei dem Raspberry Pi als auch beim Tinkerboard sind WLAN und Bluetooth Schnittstellen inbegriffen. Da aber das TinkerBoard eher dürftige Dokumentation und Unterstützung von Software besitzt \parencite[Fazit]{Finnamore2017} bevorzugen wir das Raspberry Pi.

\section{Programmiersprache}


\subsection{Informatik}
Die nachfolgende Recherche wurde unter der Annahme gemacht, dass wir in der Informatik das Raspberry Pi (nachfolgend Raspbi genannt) als Plattform benutzen werden. Generell sollte aber Vor- und Nachteile der Sprachen universell sein.

\vspace{1em}
\noindent
\begin{tabular}{|l|l|}
	\hline 
	\textbf{Beschreibung} & \textbf{Quelle} \\
	\hline
	Raspbi untersützte Sprachen& https://www.raspberrypi.org/help/faqs/\#softwareLanguages \\
	\hline
	Python Pro \& Cons & https://www.infoworld.com/article/2887974/\\
	&application-development/\\
	&a-developer-s-guide-to-the-pro-s-and-con-s-of-python.html \\ 
	\hline
	Interpreted languages: Pros \& Cons & https://stackoverflow.com/questions/1610539/\\
	&pros-and-cons-of-interpreted-languages\\
	\hline
	C++ Pros \& Cons & https://en.wikiversity.org/wiki/C\%2B\%2B\\
	\hline
\end{tabular}

\section{Bilderkennung}

\vspace{1em}
\noindent
\begin{tabular}{|l|l|}
	\hline 
	\textbf{Beschreibung} & \textbf{Quelle} \\
	\hline
	OpenCV & http://opencv.org \\
	\hline
	ImageJ & https://imagej.net/ImageJ \\
	\hline
	Fiji & http://fiji.sc/ \\
	\hline
	Eigener Algorithmus & http://szeliski.org/Book/ \\
	\hline
\end{tabular}

\section{Ziel- und Lasterkennung}

\vspace{1em}
\noindent
\begin{tabular}{|l|l|}
	\hline 
	\textbf{Beschreibung} & \textbf{Quelle} \\
	\hline
	&  \\
	\hline
\end{tabular}

\section{Stabilisierung}

\vspace{1em}
\noindent
\begin{tabular}{|l|l|}
	\hline 
	\textbf{Beschreibung} & \textbf{Quelle} \\
	\hline
	 &  \\
	\hline
\end{tabular}

\section{Antrieb \& Aufhängung}

\vspace{1em}
\noindent
\begin{tabular}{|l|l|}
	\hline 
	\textbf{Beschreibung} & \textbf{Quelle} \\
	\hline
	Aufhängung & https://de.wikipedia.org/wiki/Pendelbahn  \\
	\hline
	Test & Test \\
	\hline
\end{tabular}

\vspace{1em}
\noindent



\chapter{Lösungskonzept / Grobkonzept}
TODO Testat 2
Zusammensetzen verschiedener recherchierter Technologien zu Gesamtkonzepten. Diese Konzepte dann beurteilen und eines wählen. 

\chapter{Schlussdiskussion}

\chapter{Verzeichnisse}
TODO Abbildungs-, Tabellen-, Formel-, Quellenverzeichnisse
\printbibliography

\chapter{Anhang}
\section{Meilensteinberichte}

\section{Task-Listen}

\end{document}
